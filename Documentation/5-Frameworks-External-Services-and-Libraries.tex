\chapter{Frameworks, External Services and Libraries}

\section{Flutter}
To develop CookingTime application we decided to use Flutter platform, which allows to build cross-platform application basing on Dart programming language. 
%TODO improve description of flutter
In order to exploit all the feature of the application as a team we needed to use some other external services that could be easily imported as dependencies into out project environment. 
The usage of all that kind of services helped in terms of reducing the complexity of the implementation and to enforce non-functional requirements like security or authentication tasks.

\section{Google Firebase}
Google Firebase is a service offered by Google in order to provide some easily accessible service and to manage critical tasks like authentication in a simpler and smarter way from the perspective of the programmer.

\subsection{Authentication}
The service allowed us to have a unique container for all accounts from the different channel of authentication we offer as email, Facebook and Google account. 
Moreover it is able to manage all the operation without putting in plain text confidential information about user enforced by a SHA-1 encryption.

\subsection{Cloud Firestore}
Cloud Firestore is part of Firebase, it's basically a database with a document based structure (so NoSQL).
We need it as we have to store data about users and recipes.
Even if the model is NoSQL, we have some instruction provided by the Firebase interface to query the collection of data very similar to SQL one.
We also defined a precise set of rules for the database that restrict unauthorized users to read, create, update or delete documents if they do not have permission to do that action.
Finally, we added some secondary indexes to speed up the query done to the database when needed.

\subsection{Firebase Storage}
As the application has to store picture devoted to profile, dishes and recipe steps, the storage provided by Firestore is the best solution because we are able to access them, after uploading them, through the generated URL within the project online space. 
Of course has there are limitation of the amount of data we can upload, once we modify the pictures we delete the one no more in use.

\section{Social Media}
As mentioned before we implemented the support to authentication using Facebook or Google accounts, this is easily achievable through the API of the platforms.
Basically the APIs convert the credential of the social media account into a token which can be used and stored by Firebase. 
In addition the users are allowed to share a recipe on their preferred social media accounts.
