\chapter{Test Cases}
The testing phase of CookingTime was splitted into two steps.

\begin{itemize}
	\item \textbf{Unit test:} In this step we tested some function like validators in order to see if they are behaving in the correct way.
	\item \textbf{Integration-Widget test:} This was the last task performed, in which we automated the testing of the behaviour of widgets.
\end{itemize}

Here are presented some example od the most realvant test we made:

\section{Unit test}
%TODO 1 table of validators test

\section{Integration-Widget test}
This type of testing was automated for all the feature that are repeatable over the time, all the action non reversible were tested manually
%TODO integration test tables

\begin{table}[H]
	\centering
	\begin{tabular}{|l|l|}
	\hline
	\textbf{Test Case}& Successful Login\\
	\hline
	\textbf{Goal}& Login a user\\
	\hline
	\textbf{Input}& 
	\begin{minpage}{.7\linewidth}
	In the login screen the user provide a valid email and password. In the end he taps on login button.
	\end{minipage}\\
	\hline
	\textbf{Expected Outcome}& The user is logged in the and he's routed to the home page.\\
	\hline
	\textbf{Actual Outcome}& 
	\begin{minpage}{.7\linewidth}
	CORRECT: The application allows the user to provide his credential, then tapping the login button the user is logged in the application.
	\end{minipage}\\
	\hline	
	\end{tabular}
	\caption{Successful Login Test}
\end{table}

\begin{table}[H]
	\centering
	\begin{tabular}{|l|l|}
	\hline
	\textbf{Test Case}& Successful Registration\\
	\hline
	\textbf{Goal}& Register a new user\\
	\hline
	\textbf{Input}& 
	\begin{minpage}{.7\linewidth}
	In the login screen the user tap on the regiresting button, then in the new page provide a valid username, email and password. In the end he taps on register button.
	\end{minipage}\\
	\hline
	\textbf{Expected Outcome}& 
	\begin{minpage}{.7\linewidth}
	The new user is registred and the application routes the user to the home page.
	\end{minipage}\\
	\hline
	\textbf{Actual Outcome}& 
	\begin{minpage}{.7\linewidth}
	CORRECT: The application provide the right screen after clicking on the register button and allows the user to provide his credential, then tapping the registration button the user is registered and logged in the application.
	\end{minipage}\\
	\hline	
	\end{tabular}
	\caption{Successful Registration Test}
\end{table}

\begin{table}[H]
	\centering
	\begin{tabular}{|l|l|}
	\hline
	\textbf{Test Case}& Failed Registration\\
	\hline
	\textbf{Goal}& Try to register a new user omitting some required information\\
	\hline
	\textbf{Input}& 
	\begin{minpage}{.7\linewidth}
	In the login screen the user tap on the regiresting button, then in the new page provide one or two among valid username, email and password. In the end he taps on register button.
	\end{minipage}\\
	\hline
	\textbf{Expected Outcome}&
	\begin{minpage}{.7\linewidth}
	The new user is not registred and tha application throw some alert on the screen.
	\end{minipage}\\
	\hline
	\textbf{Actual Outcome}& 
	\begin{minpage}{.7\linewidth}
	CORRECT: The application provide the right screen after clicking on the register button and allows the user to provide his credential, then tapping the registration button the user is not registered and alerts are thrown.
	\end{minipage}\\
	\hline	
	\end{tabular}
	\caption{Failed Registration Test}
\end{table}

\begin{table}[H]
	\centering
	\begin{tabular}{|l|l|}
	\hline
	\textbf{Test Case}& Successful Password Reset\\
	\hline
	\textbf{Goal}& Reset the password of a user who lost it\\
	\hline
	\textbf{Input}& 
	\begin{minpage}{.7\linewidth}
	In the login screen the user tap on the forgot password button, then in the new page provide email of the account. In the end he taps on send reset email button.
	\end{minipage}\\
	\hline
	\textbf{Expected Outcome}& The user get a mail to reset his password.\\
	\hline
	\textbf{Actual Outcome}& 
	\begin{minpage}{.7\linewidth}
	CORRECT: The application provide the right screen after clicking on the forgot password button and allows the user to provide his email, then tapping the send reset email button the user recieves the reset email.
	\end{minipage}\\
	\hline	
	\end{tabular}
	\caption{Successful Password Reset Test}
\end{table}

\begin{table}[H]
	\centering
	\begin{tabular}{|l|l|}
	\hline
	\textbf{Test Case}& Failed Password Reset\\
	\hline
	\textbf{Goal}& Try to reset the password of a user who lost it omitting the email\\
	\hline
	\textbf{Input}& 
	\begin{minpage}{.7\linewidth}
	In the login screen the user tap on the forgot password button, then he taps on send reset email button without providing the email.
	\end{minipage}\\
	\hline
	\textbf{Expected Outcome}& The user doesn't get a mail to reset his password.\\
	\hline
	\textbf{Actual Outcome}& 
	\begin{minpage}{.7\linewidth}
	CORRECT: The application provide the right screen after clicking on the forgot password button and then tapping the send reset email button the user doesn't recieve the reset email.
	\end{minipage}\\
	\hline	
	\end{tabular}
	\caption{Failed Password Reset Test}
\end{table}

\begin{table}[H]
	\centering
	\begin{tabular}{|l|l|}
	\hline
	\textbf{Test Case}& Filter Recipes\\
	\hline
	\textbf{Goal}& 
	\begin{minpage}{.7\linewidth}
	The user goes in the search screen and select some parameter to filter the recipes and the ones listed out are retrieved according to the specified parameters.
	\end{minipage}\\
	\hline
	\textbf{Input}& 
	\begin{minpage}{.7\linewidth}
	In the home screen the user tap on the search button on the bottom menù, then he taps the filter button and select some parameters.
	\end{minipage}\\
	\hline
	\textbf{Expected Outcome}& 
	\begin{minpage}{.7\linewidth}
	The application shows only recipes according the parameters selected by the user.
	\end{minipage}\\
	\hline
	\textbf{Actual Outcome}& 
	\begin{minpage}{.7\linewidth}
	CORRECT: The application provide the right screen after clicking on the search button and tapping the filter the user in able to select some parameter in order to filter the recipes, after filtering the retrieved recipes are quieried according to the parameter specified before.
	\end{minipage}\\
	\hline	
	\end{tabular}
	\caption{Filter Recipes Test}
\end{table}

\begin{table}[H]
	\centering
	\begin{tabular}{|l|l|}
	\hline
	\textbf{Test Case}& Show User information and recipes\\
	\hline
	\textbf{Goal}& See correct user data and recipes\\
	\hline
	\textbf{Input}& In the home screen the user taps the account button in the bottom menù.\\
	\hline
	\textbf{Expected Outcome}& 
	\begin{minpage}{.7\linewidth}
	The user is able to see his personal information, his statistics and his recipes.
	\end{minipage}\\
	\hline
	\textbf{Actual Outcome}&
	\begin{minpage}{.7\linewidth}
	CORRECT: The application allows the user to see the profile picture, username, written recipes and received reviews after tapping the account button in the home page.
	\end{minipage}\\
	\hline	
	\end{tabular}
	\caption{Show User information and recipes}
\end{table}

\begin{table}[H]
	\centering
	\begin{tabular}{|l|l|}
	\hline
	\textbf{Test Case}& Change user information\\
	\hline
	\textbf{Goal}& Modify profile picture, username and password\\
	\hline
	\textbf{Input}& 
	\begin{minpage}{.7\linewidth}
	In the home screen the user taps the account button in the bottom menù, then he goes in the setting page through the setting button and insert new information and press the save button.
	\end{minipage}\\
	\hline
	\textbf{Expected Outcome}& The user is able to see his updated personal information.\\
	\hline
	\textbf{Actual Outcome}& 
	\begin{minpage}{.7\linewidth}
	CORRECT: The application allows the user to set a new profile picture, username and password. After tapping the save button he is able to see the modified information updated.
	\end{minipage\\
	\hline	
	\end{tabular}
	\caption{Change user information}
\end{table}