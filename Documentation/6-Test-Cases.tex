\chapter{Test Cases}
The testing phase of CookingTime was split into two steps.

\begin{itemize}
	\item \textbf{Unit test:} In this step we tested some function, like text validators, in order to see if they are behaving in the correct way.
	\item \textbf{Integration-Widget test:} Since all the functionalities of the application are strictly tied with Firebase, we skipped entirely the Widget testing phase and we did a deep integration testing.
	This was the last task performed, in which we automated the testing of the behaviour across multiple widgets and Firebase's services.
\end{itemize}

Here are presented some example of the most relevant test we made:

\section{Unit test}
%TODO 1 table of validators test

\section{Integration-Widget test}
This type of testing was automated for all the feature that are repeatable over the time, while all the non reversible actions were tested manually.
%TODO integration test tables

\begin{table}[H]
	\centering
	\begin{tabular}{|l|l|}
	\hline
	\textbf{Test Case}& Successful Login\\
	\hline
	\textbf{Goal}& Login a user\\
	\hline
	\textbf{Input}& 
	\begin{minipage}{.7\linewidth}
	In the login screen the user provides a valid email and password. In the end he taps on login button.
	\end{minipage}\\
	\hline
	\textbf{Expected Outcome}& The user is logged in the and he's routed to the home page.\\
	\hline
	\textbf{Actual Outcome}& 
	\begin{minipage}{.7\linewidth}
	CORRECT: The application allows the user to provide his credential, then tapping the login button the user is logged in the application.
	\end{minipage}\\
	\hline	
	\end{tabular}
	\caption{Successful Login Test}
\end{table}

\begin{table}[H]
	\centering
	\begin{tabular}{|l|l|}
	\hline
	\textbf{Test Case}& Successful Registration\\
	\hline
	\textbf{Goal}& Register a new user\\
	\hline
	\textbf{Input}& 
	\begin{minipage}{.7\linewidth}
	In the login screen the user tap on the registering button, then in the new page provide a valid username, email and password. In the end he taps on register button.
	\end{minipage}\\
	\hline
	\textbf{Expected Outcome}& 
	\begin{minipage}{.7\linewidth}
	The new user is registered and the application routes the user to the home page.
	\end{minipage}\\
	\hline
	\textbf{Actual Outcome}& 
	\begin{minipage}{.7\linewidth}
	CORRECT: The application provides the right screen after clicking on the register button and allows the user to provide his credential, then tapping the registration button the user is registered and logged in the application.
	\end{minipage}\\
	\hline	
	\end{tabular}
	\caption{Successful Registration Test}
\end{table}

\begin{table}[H]
	\centering
	\begin{tabular}{|l|l|}
	\hline
	\textbf{Test Case}& Failed Registration\\
	\hline
	\textbf{Goal}& Try to register a new user omitting some required information\\
	\hline
	\textbf{Input}& 
	\begin{minipage}{.7\linewidth}
	In the login screen the user taps on the registration button, then in the new page provides one or two among valid username, email and password. In the end he taps on register button.
	\end{minipage}\\
	\hline
	\textbf{Expected Outcome}&
	\begin{minipage}{.7\linewidth}
	The new user is not registered and the application throws some alert on the screen.
	\end{minipage}\\
	\hline
	\textbf{Actual Outcome}& 
	\begin{minipage}{.7\linewidth}
	CORRECT: The application provides the right screen after clicking on the register button and allows the user to provide his credential, then tapping the registration button the user is not registered and alerts are thrown.
	\end{minipage}\\
	\hline	
	\end{tabular}
	\caption{Failed Registration Test}
\end{table}

\begin{table}[H]
	\centering
	\begin{tabular}{|l|l|}
	\hline
	\textbf{Test Case}& Successful Password Reset\\
	\hline
	\textbf{Goal}& Reset the password of a user who lost it\\
	\hline
	\textbf{Input}& 
	\begin{minipage}{.7\linewidth}
	In the login screen the user tap on the forgot password button, then in the new page provide email of the account. In the end he taps on send reset email button.
	\end{minipage}\\
	\hline
	\textbf{Expected Outcome}& The user gets a mail to reset his password.\\
	\hline
	\textbf{Actual Outcome}& 
	\begin{minipage}{.7\linewidth}
	CORRECT: The application provides the right screen after clicking on the forgot password button and allows the user to provide his email, then tapping the send reset email button the user receives the reset email.
	\end{minipage}\\
	\hline	
	\end{tabular}
	\caption{Successful Password Reset Test}
\end{table}

\begin{table}[H]
	\centering
	\begin{tabular}{|l|l|}
	\hline
	\textbf{Test Case}& Failed Password Reset\\
	\hline
	\textbf{Goal}& Try to reset the password of a user who lost it omitting the email\\
	\hline
	\textbf{Input}& 
	\begin{minipage}{.7\linewidth}
	In the login screen the user taps on the forgot password button, then he taps on send reset email button without providing the email.
	\end{minipage}\\
	\hline
	\textbf{Expected Outcome}& The user doesn't get a mail to reset his password.\\
	\hline
	\textbf{Actual Outcome}& 
	\begin{minipage}{.7\linewidth}
	CORRECT: The application provides the right screen after clicking on the forgot password button and then tapping the send reset email button the user doesn't receive the reset email.
	\end{minipage}\\
	\hline	
	\end{tabular}
	\caption{Failed Password Reset Test}
\end{table}

\begin{table}[H]
	\centering
	\begin{tabular}{|l|l|}
		\hline
		\textbf{Test Case}& Correct Visualization of a Recipe\\
		\hline
		\textbf{Goal}& Check the correct loading and visualization of a recipe\\
		\hline
		\textbf{Input}& 
		\begin{minipage}{.7\linewidth}
			In the homepage screen the user taps on the first recipe card available and scroll down to the bottom.
		\end{minipage}\\
		\hline
		\textbf{Expected Outcome}& All the elements of the recipe are correctly loaded and visualized.\\
		\hline
		\textbf{Actual Outcome}& 
		\begin{minipage}{.7\linewidth}
			CORRECT: The application provides the right screen after clicking on the recipe card button and then while scrolling through the bottom all widget that display the recipe attributes are displayed.
		\end{minipage}\\
		\hline	
	\end{tabular}
	\caption{Correct Visualization of a Recipe Test}
\end{table}

\begin{table}[H]
	\centering
	\begin{tabular}{|l|l|}
		\hline
		\textbf{Test Case}& Successful Writing Updating and Deleting of a Recipe\\
		\hline
		\textbf{Goal}&Create, visualize, update and delete a recipe\\
		\hline
		\textbf{Input}& 
		\begin{minipage}{.7\linewidth}
			The user goes in the write a recipe screen by tapping the write button on the bottom menu, insert all the information necessary and submit the recipe. Than he verify the presence of the recipe in the user's profile, from there he first modify it and then delete it.
		\end{minipage}\\
		\hline
		\textbf{Expected Outcome}& The user can create, visualize, update and delete a recipe.\\
		\hline
		\textbf{Actual Outcome}& 
		\begin{minipage}{.7\linewidth}
			CORRECT: The application allows the user to submit the recipe only when all the required data is inputted. Then the user can correctly visualize his recipe in both the homepage and user profile page, from which he can access it, modify it and delete it. After the deletion the recipe is correctly not visible anymore.
		\end{minipage}\\
		\hline	
	\end{tabular}
	\caption{Successful Writing Updating and Deleting of a Recipe Test}
\end{table}

\begin{table}[H]
	\centering
	\begin{tabular}{|l|l|}
		\hline
		\textbf{Test Case}& Successful Writing and Deleting of a Review\\
		\hline
		\textbf{Goal}& Crete, visualize and delete a review\\
		\hline
		\textbf{Input}& 
		\begin{minipage}{.7\linewidth}
			From the home screen the user checks the first recipe available and scroll up to to the review form. He then write a review with rating and comment and submit it. After the review is correctly visualized the user deletes the review.
		\end{minipage}\\
		\hline
		\textbf{Expected Outcome}& The user can properly create, visualize and delete their review.\\
		\hline
		\textbf{Actual Outcome}& 
		\begin{minipage}{.7\linewidth}
			CORRECT: The application allows the user to reach and write a review under a recipe. Then the user can correctly visualize his review and can correctly delete his review after a confirmation prompt. The review is then correctly no long available.
		\end{minipage}\\
		\hline	
	\end{tabular}
	\caption{Successful Writing and Deleting of a Review Test}
\end{table}

\begin{table}[H]
	\centering
	\begin{tabular}{|l|l|}
	\hline
	\textbf{Test Case}& Filter Recipes\\
	\hline
	\textbf{Goal}& 
	\begin{minipage}{.7\linewidth}
	The user goes in the search screen and select some parameter to filter the recipes and the ones listed out are retrieved according to the specified parameters.
	\end{minipage}\\
	\hline
	\textbf{Input}& 
	\begin{minipage}{.7\linewidth}
	In the home screen the user taps on the search button on the bottom menu, then he taps the filter button and select some parameters.
	\end{minipage}\\
	\hline
	\textbf{Expected Outcome}& 
	\begin{minipage}{.7\linewidth}
	The application shows only recipes according the parameters selected by the user.
	\end{minipage}\\
	\hline
	\textbf{Actual Outcome}& 
	\begin{minipage}{.7\linewidth}
	CORRECT: The application provides the right screen after clicking on the search button and tapping the filter the user in able to select some parameters in order to filter the recipes, after filtering the retrieved recipes are queried according to the parameter specified before.
	\end{minipage}\\
	\hline	
	\end{tabular}
	\caption{Filter Recipes Test}
\end{table}

\begin{table}[H]
	\centering
	\begin{tabular}{|l|l|}
		\hline
		\textbf{Test Case}& Show User Saved Recipes\\
		\hline
		\textbf{Goal}& See saved recipes in the saved recipe view\\
		\hline
		\textbf{Input}& 
		\begin{minipage}{.7\linewidth}
			From the homepage screen the user checks that the saved recipes view is empty. Then from the latest recipes view he saves the first two recipes and checks them in the saved recipes view, where they should be present. He then un-saves one of the two and controls that only the remaining one is still visible. He finally un-saves the other recipe and checks that there no saved recipes anymore.
		\end{minipage}\\
		\hline
		\textbf{Expected Outcome}&
		\begin{minipage}{.7\linewidth}
		The user can properly add, visualize and remove recipes form the saved recipes view.
		\end{minipage}\\
		\hline
		\textbf{Actual Outcome}& 
		\begin{minipage}{.7\linewidth}
			CORRECT: The application allows the user to save and visualize saved recipes in the correct widget on the homepage. He can then add further recipes and remove old one, after which they are no longer visualized.
		\end{minipage}\\
		\hline	
	\end{tabular}
	\caption{Show User Saved Recipes Test}
\end{table}

\begin{table}[H]
	\centering
	\begin{tabular}{|l|l|}
	\hline
	\textbf{Test Case}& Show User information and recipes\\
	\hline
	\textbf{Goal}& See correct user data and recipes\\
	\hline
	\textbf{Input}& 
	\begin{minipage}{.7\linewidth}
	In the home screen the user taps the account button in the bottom menu.
	\end{minipage}\\
	\hline
	\textbf{Expected Outcome}& 
	\begin{minipage}{.7\linewidth}
	The user is able to see his personal information, his statistics and his recipes.
	\end{minipage}\\
	\hline
	\textbf{Actual Outcome}&
	\begin{minipage}{.7\linewidth}
	CORRECT: The application allows the user to see the profile picture, username, written recipes and received reviews after tapping the account button in the home page.
	\end{minipage}\\
	\hline	
	\end{tabular}
	\caption{Show User information and recipes}
\end{table}

\begin{table}[H]
	\centering
	\begin{tabular}{|l|l|}
	\hline
	\textbf{Test Case}& Change user information\\
	\hline
	\textbf{Goal}& Modify profile picture, username and password\\
	\hline
	\textbf{Input}& 
	\begin{minipage}{.7\linewidth}
	In the home screen the user taps the account button in the bottom menu, then he goes in the setting page through the setting button and inserts new information and presses the save button.
	\end{minipage}\\
	\hline
	\textbf{Expected Outcome}& The user is able to see his updated personal information.\\
	\hline
	\textbf{Actual Outcome}& 
	\begin{minipage}{.7\linewidth}
	CORRECT: The application allows the user to set a new profile picture, username and password. After tapping the save button he is able to see the modified information updated.
	\end{minipage}\\
	\hline	
	\end{tabular}
	\caption{Change user information}
\end{table}