\chapter{General Overview}

\section{Concept}
	CookingTime is a multi-platform application that allows users to share their homemade cooking recipes with others and to search for new ones among the variety of recipes posted by other people.
	Furthermore, users can save their favorite recipes in a personal recipe book in the app in order to always have them at hand and never lose them.
	Users will also be able to give suggestions by writing reviews and rate recipes of other users.
	To access these functionalities, users are required to register to the server via mail and password, Google account or Facebook account.
	The main components of the project are:
	\begin{itemize}
		\item The client side part, the application itself, which is the first point of interaction with the user.
		It allows the user to access all the functionalities of the app, the app will then automatically query the server-side database based on the user's requests.
		
		\item The server side part, fully made with the Firebase platform, contains the relational database used to store all the users' data, which comprehend user's login information, recipes and reviews.
	\end{itemize}


\section{Core features}
	This section of the document is devoted to present the main functionalities provided by CookingTime, to let the reader understand better these features, 
	they are divided by screen in which they are implemented.
	
	\subsection{Authentication Screen}
		\begin{itemize}
			\item Allows the user to register his/her-self in the application.
		
			\item Allows the user to authenticate using his/her email and password.
			
			\item Allows the user to authenticate using his/her Facebook account.
			
			\item Allows the user to authenticate using his/her Google account.
		\end{itemize}

	\subsection{Home Screen}
		\begin{itemize}
			\item Allows the user to see the latest posted recipes from all the other users.
			
		\end{itemize}

	\subsection{Write a Recipe Screen}
		\begin{itemize}
			\item Allows the user to post a recipe.
			
			\item Allows the user to upload a picture representing the recipe.
			
			\item Allows the user to insert a description, category, difficulty and preparation time of the recipe.
			
			\item Allows the user to add up to 10 steps for the preparation process of the recipe, each one including title, description and a photo.
		\end{itemize}

	\subsection{Search for Recipes Screen}
		\begin{itemize}
			\item Allows the user to filter recipes depending on categories, difficulty, rating and preparation time.
		\end{itemize}

	\subsection{Write a Review Screen}
		\begin{itemize}
			\item Allows the user to insert a rating and a text review of a recipe.
		\end{itemize}

	\subsection{Saved Recipes List Screen}
		\begin{itemize}
			\item Allows the user to see all his/her saved recipes.
		\end{itemize}

	\subsection{Account Settings Screen}
		\begin{itemize}
			\item Allows the user to modify his/her personal information.
			
			\item Allows the user to see all his/her posted recipes.
			
			\item Allows the user to see his/her personal statistics.
			
			\item Allows the user to sign out from his/her account.
		\end{itemize}


\section{Functional Requirements}
	In this section we present the functional requirements necessary to the correct behavior of the system.

	\subsection{General Requirements}
		\begin{itemize}
			\item The application should be comprehensible by as many people as possible, so we decided to use the English language.
		\end{itemize}

	\subsection{Authentication Requirements}
		\begin{itemize}
			\item The \textit{Login screen} should be accessible only to users not currently logged in.
			\item The \textit{Login screen} should allow the user to login using email and password, or Facebook, or Google.
			\item The \textit{Register screen} should allow users to register using email, password and providing a username. 
			\item \textit{Authentication screens} should redirect the user to the Home screen if the authentication is successful.
		\end{itemize}

	\subsection{Home Requirements}
		\begin{itemize}
			\item The \textit{Home screen} should provide access to the user to all the application functionalities.
			\item The \textit{Home screen} should show to the user the latest recipes published by all the users.
			\item The \textit{Home screen} should provide access to the \textit{Recipe screens} by pressing on a specific recipe.
		\end{itemize}

	\subsection{Recipe View Requirements}
	\begin{itemize}
		\item The \textit{Recipe screen} should all information about the requested recipe, which comprehend: title, subtitle, author, description, image, rating, difficulty, category, intolerance, servings, time required, ingredients, steps and users' reviews.
		\item The \textit{Recipe screen} should also provide the \textit{Write a Review widget}.
		\item The \textit{Recipe screen} should allow the user to save the recipe in his saved recipes list.
	\end{itemize}

	\subsection{Write a Recipe Requirements}
		\begin{itemize}
			\item The \textit{Write a Recipe screen} should allow the user to provide all the necessary information needed to publish a recipe.
			\item The \textit{Write a Recipe screen} should allow the user to add images from both the gallery and the camera.
			\item The \textit{Write a Recipe scree}n should tell the user if some information is missing.
			\item After a recipe has been submitted it should be shown at the top of the \textit{Home screen}.
		\end{itemize}

	\subsection{Search for Recipes Requirements}
		\begin{itemize}
			\item The \textit{Search a Recipe screen} should allow all user to filter the recipes in the database based on difficulty, category, preparation time and intolerance.
			\item The \textit{Search a Recipe screen} should allow all user to sort the recipes in the database based on rating or submission time.
		\end{itemize}

	\subsection{Review Requirements}
		\begin{itemize}
			\item The \textit{Reviews widget} should allow the user to read the most recent reviews about the current recipe in the screen.
			\item The \textit{Write a Review widget} should allow the user to rate and write a review about the current recipe in the screen.
			\item The \textit{Write a Review widget} should forbid the user to write a second review for the same recipe.
			\item The \textit{Write a Review widget} should forbid the author of the recipe to write a review for that recipe.
		\end{itemize}

	\subsection{Saved Recipes List Requirements}
		\begin{itemize}
			\item The \textit{Saved Recipes screen} should allow the user to access all the recipes previously saved.
			\item The \textit{Saved Recipes screen} should allow the user to remove some of the saved recipes.
		\end{itemize}

	\subsection{Account Settings Requirements}
		\begin{itemize}
			\item The \textit{Settings screen} should allow the user to change his username, profile image and password.
			\item The \textit{Setting screen} should allow the user to log out from his account.
			\item The \textit{Setting screen} should display the user's total recipes, reviews received, rating among his recipes and latest recipes written.
		\end{itemize}


\section{Non-Functional Requirements}
	These are the non-functional requirements that are necessary to guarantee a functional application:

\begin{itemize}
	\item \textbf{Portability}: Using a cross-platform implementation allows the application to run both in Android and iOS systems. The application could be run on different sized devices as well.
	
	\item \textbf{Usability}: All the features of the application are quite easy to be found as the design of the interfaces are very user-friendly in order to provide the best user experience.
	
	\item \textbf{Availability and Reliability}: The application must be always working and be connected to the Firebase platform.
	The downtime is strictly related to the one of Firebase, which is very low.
	In case of any kind of failure the application must be restored to a correct state as soon as possible.
	
	\item \textbf {Security}: As the user have different way to sign in the application and they are providing some personal data, the app must implement some security actions. For this reason, the connection between the application and the Firebase server are done using and encrypted communication through SHA-1 keys.
	
	\item \textbf{Extensibility}: The application is developed for some basic feature which can be increased over the time.
	
	\item \textbf{Maintainability}: The application was developed with a clear coding style by commenting all the part in order to let the code be readable to anyone who wants to touch it.
	
	\item \textbf{Nice User Interface}: It's important that user experience will be the best, for this reason the UI is very nice and simple.
\end{itemize}
