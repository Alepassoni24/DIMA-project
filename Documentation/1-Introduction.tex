\chapter{Introduction}

	\section{Purpose}
		The application has an academic purpose and in particular it is the project to grade the \textit{Design and Implementation of Mobile Application} master level degree course taught by professor Luciano Baresi at Politecnico di Milano.
		The project assignment asks to design and implement a mobile application using one of the four technologies presented during the lectures, the platform can be chosen by the students. 

		This document illustrates all the design and implementation decision related to CookingTime app made by Nicola Camillucci and Alessandro Passoni. 
		It provides as well a guidance to implement all the features of the app according to the decided design, and also to present the project to people not involved into the development process.


	\section{Scope}
		The aim of this application is to allow people to share their own homemade recipes and to search for new ones among the variety of recipes posted by other people, in this way users will be able to share their tradition but also to find new way to cook any kind of dishes.
		Selecting a new recipe will be helped by ratings and reviews of people who have already cooked that recipes and if they want, they can also save it in a personal digital recipe book section.
		These are obtained thank to a database which can filter and suggest the right recipes to accomplish the request of the users.


	\section{Stakeholders}
		The main stakeholder of the project are professor Luciano Baresi and his teaching assistant Giovanni Quattrocchi that are going to evaluate the application.
		Moreover users interested in this application can be considered stakeholders as well.


	\section{Time Constraints}
		There were no defined deadlines, except for the final delivery of the project on one of the exam date of the academic year.
		However the developers decided to split the work into three milestone deadlines which are on a generic and flexible date. 
		The aim is to have a sort of check points, usually named as alpha, beta and final version, in order to have some deliverable part incrementally.
		This will help to not encounter some issues about parts developed at the beginning of the implementation during the further steps of the development.
		The implementation started in the second half of November 2020 and it's planned to end by the first half of February 2021.


	\section{Risk Analysis}
		In the first phase of the development process the team identified the possible risks related to the implementation of this type of application, in order to train ourselves how to deal and avoid all these kind of issues.
		The most critical feature to deal with is the interaction with the database and the authentication phase.


	\section{Overview}
		The document is structured as follows:
		TODO
		\begin{itemize}
			\item \textbf{Section 1: Introduction.} A general introduction of the Design Document. The objective is to explain what the document is going to address.

			\item \textbf {Section 2: General Overview.} A general overview of the project. In this section, the reader can find the main features of the application and
			the requirements of the application.

			\item \textbf{Section 3: Architectural Design.} This section contains an overview of the high-level components of the system and then a more detailed 
			description of *******************. Finally, it shows the Component Interfaces and the chosen architecture styles and patterns. And also database structure?????
			
			\item \textbf{Section 4: User Interface Design.} This section contains the screenshots of the application with some descriptions of the user interfaces 
			and how to move through the application screens.
			
			\item \textbf{Section 5: Frameworks, External services and Libraries.} This section aims to explain the main frameworks, external services and libraries used 
			pointing out their advantages and possible disadvantages and why the team decided to use them.
			
			\item \textbf{Section 6: Test Case.} This section identifies how the application will be tested and shows a list test cases performed reporting their results.
			
			\item \textbf{Section 7: Effort and Cost Estimation.} A summary of worked time and cost estimation of the work.
			
			\item \textbf{Section 8: Future work.} This section presents some possible new feature which could be added further in time.
		\end{itemize}